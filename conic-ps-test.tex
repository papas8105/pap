\documentclass[a4paper,10pt]{article}
%\documentclass[a4paper,10pt]{scrartcl}

\usepackage{geometry}
 \geometry{
 a4paper,
 total={170mm,257mm},
 left=20mm,
 top=20mm,
 }
\newcommand{\ttt}{\texttt}
\setlength{\parindent}{0em}
\usepackage{xltxtra}
\usepackage{kmath}
\setromanfont[Mapping=tex-text]{Linux Libertine O}
\usepackage{pstricks-add}
\usepackage{epsfig}
\begin{document}
\textbf{A conical glass is filled with water at a steady rate, prove that the function of the height of the surface of the water in the glass with 
respect to time has a cube root behavior.}
\[
\newrgbcolor{ffqqtt}{1 0 0.2}
\newrgbcolor{zzffff}{0.6 1 1}
\newrgbcolor{ttttff}{0.2 0.2 1}
\psset{xunit=0.6610169491525423cm,yunit=0.5789473684210527cm,algebraic=true,dotstyle=o,dotsize=3pt 0,linewidth=0.8pt,arrowsize=3pt 2,arrowinset=0.25}
\begin{pspicture*}(-5.76,-1.2)(6.04,8.3)
\pspolygon[linewidth=2pt,linecolor=zzffff,fillcolor=zzffff,fillstyle=solid,opacity=0.25](-5,7)(0,7)(0,0)
\pspolygon[linewidth=2pt,linecolor=yellow,fillcolor=yellow,fillstyle=solid,opacity=0.25](0,0)(0.84,3.04)(0,4)
\psline(0,0)(-5,7)
\psline(0,0)(5,7)
\rput{-0.54}(0.03,6.89){\psellipse(0,0)(5.05,0.89)}
\psline[linewidth=2pt,linecolor=ffqqtt](0,7)(0,0)
\rput[tl](-4.94,4.18){$$ \texttt{h} $$}
\rput{-179.99}(0.04,4){\psellipse[linewidth=2pt,linecolor=ffqqtt,fillcolor=ffqqtt,fillstyle=solid,opacity=0.25](0,0)(2.88,1)}
\psline[linewidth=2pt,linestyle=dashed,dash=3pt 3pt,linecolor=zzffff](-5,7)(-5,0)
\psline[linewidth=2pt,linecolor=ffqqtt](3.24,3.68)(3.24,3.98)
\rput[tl](3.34,4.22){$\texttt{d}h$}
\psline[linewidth=2pt,linestyle=dashed,dash=3pt 3pt,linecolor=red](0,7)(-5,7)
\rput[tl](-2.34,7.68){$\texttt{R}$}
\psline[linewidth=2pt,linestyle=dashed,dash=3pt 3pt,linecolor=ttttff](0,4)(0.84,3.04)
\rput[tl](0.46,4.14){$r(t)$}
\rput[tl](4.06,2.74){$h(t)$}
\psline[linewidth=2pt,linecolor=zzffff](-5,7)(0,7)
\psline[linewidth=2pt,linecolor=zzffff](0,7)(0,0)
\psline[linewidth=2pt,linecolor=zzffff](0,0)(-5,7)
\psline[linewidth=2pt,linecolor=yellow](0,0)(0.84,3.04)
\psline[linewidth=2pt,linecolor=yellow](0.84,3.04)(0,4)
\psline[linewidth=2pt,linecolor=yellow](0,4)(0,0)
\psline[linewidth=2pt,linestyle=dashed,dash=3pt 3pt,linecolor=yellow](4,4)(4,0)
\end{pspicture*}
\]
From the similarity of the cyan and the yellow triangles we get 
\[\frac{R}{h} = \frac{r(t)}{h(t)}\Rightarrow r(t) = c\cdot h(t)\]
the infinitesimal volume of water added is
\[
 \frac{\ttt{d}V}{\ttt{d}t} = \pi\cdot r(t)^{2}\cdot \frac{\ttt{d}h}{\ttt{d}{t}}\Rightarrow c_{1} = \pi\cdot r(t)^{2}\cdot \frac{\ttt{d}h}{\ttt{d}{t}}
\]
with substitution we get
\begin{eqnarray*}
 c_1 & = & \pi\cdot c^2\cdot h(t)^{2} \cdot \frac{\ttt{d}h}{\ttt{d}t}\\
 h(t)^{2}\ttt{d}h & = & c_{1} \frac{\ttt{d}t}{c^{2} \pi}\\
 \int h(t)^{2}\;\ttt{d}h & = & \frac{c_{1}}{c^{2}\pi}\int1\;\ttt{d}t\\
 \frac{h(t)^{3}}{3} & = & \frac{c_{1}}{c^{2}\pi}t + c_2\\
 h(t)^{3} & = & 3\frac{c_{1}}{c^{2}\pi}t + 3c_2\\
 h(t)  & = & \sqrt[3]{3\frac{c_{1}}{c^{2}\pi}t + 3c_2}
\end{eqnarray*}
.
\end{document}
